%  Template erstellt von Felix Kr�ger am 18.10.2012.
%  Copyright (c) 2012 TUB_Physik. All rights reserved.
%
\documentclass[a4paper,ngerman]{article}
\usepackage[ngerman]{babel} % erm�glicht deutsche Silbentrennung und direkte Eingabe von Umlauten
\usepackage[a4paper, left=2.5cm, right=2cm, top=1.5cm, bottom=2cm]{geometry}
\usepackage[ansinew]{inputenc} % teilt LaTeX die Texcodierung mit. Bei Mac OS X: applemac Bei Windowssystemen: ansinew
\usepackage[T1]{fontenc} % erm�glicht die Silbentrennung von W�rtern mit Umlauten
\usepackage{amsmath,amssymb}    % Mathesymbole
\usepackage{boxedminipage}      % erm�glich die Box um Grafiken und Textbereiche




\begin{document}


\noindent \Large{\underline{\textbf{Physik-Bachelor-Modulpr�fung}}} \\


% In der folgenden Zeile \bigotimes f�r eine Kreis mit Kreuz drin, und \bigcirc nur f�r den Kreis. Im Wahlbereich ggf das Fach angeben. 
$\bigcirc$ \,Exp \qquad $\bigcirc$ \,Theo 1 \qquad $\bigotimes$ \,Mathe 1/2 \qquad $\bigcirc$ \, Wahlpflicht: \emph{}\\
\\


% In den folgenden Zeilen wird der Kopf definiert. Sollte sich selbsterkl�ren.
\begin{boxedminipage}{\linewidth}
\begin{tabular}{ll}
\textbf{Pr�fer:} \emph{...}        		   &
\textbf{Beisitzer:} \emph{...}   			\\
\textbf{Datum:} \emph{...}     				&
\textbf{Note:} \emph{...}						\\
\textbf{Anzahl der Kandidaten:} \emph{1}&
\textbf{Dauer:} \emph{...}  \\
\end{tabular}
\end{boxedminipage}\\
\\
\textbf{Vorbereitungszeit:} \emph{...} \\
\textbf{B�cher:} \emph{...} \\



% In den folgenden Zeilen soll die Pr�fung kurz zusammengefasst werden. Als Liste wie vorgeschlagen oder als zusammenh�ngender Text. Wenn das Protokoll zu lang wird, einfach die boxed minipage rausnehmen.
\begin{boxedminipage}{\linewidth}
\textbf{\underline{Pr�fung:}}


\begin{itemize}
\item 
\item 
\item 

\end{itemize}
\end{boxedminipage}

\vspace{1ex}

% In den folgenden Zeilen kann eine eigene Bewerung �ber den Pr�fer und den Ablauf der Pr�fung abgegeben werden.
\textbf{\underline{Beurteilung der Pr�fung und der Pr�fer:}}\\
\emph{ }

\end{document}
